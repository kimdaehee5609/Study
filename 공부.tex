%	-------------------------------------------------------------------------------
% 
%		공부
%
%		2022년
%		08월
%		14일
%		토요일
%%		첫작업
%
%
%
%
%
%
%
%	-------------------------------------------------------------------------------

%	\documentclass[12pt, a3paper, oneside]{book}
	\documentclass[12pt, a4paper, oneside]{book}
%	\documentclass[12pt, a4paper, landscape, oneside]{book}

		% --------------------------------- 페이지 스타일 지정
		\usepackage{geometry}
%		\geometry{landscape=true	}
		\geometry{top 			=10em}
		\geometry{bottom			=10em}
		\geometry{left			=8em}
		\geometry{right			=8em}
		\geometry{headheight		=4em} % 머리말 설치 높이
		\geometry{headsep		=2em} % 머리말의 본문과의 띠우기 크기
		\geometry{footskip		=4em} % 꼬리말의 본문과의 띠우기 크기
% 		\geometry{showframe}
	
%		paperwidth 	= left + width + right (1)
%		paperheight 	= top + height + bottom (2)
%		width 		= textwidth (+ marginparsep + marginparwidth) (3)
%		height 		= textheight (+ headheight + headsep + footskip) (4)	



		%	=================================================================== package
		%	package
		%	===================================================================
%			\usepackage[hangul]{kotex}				% 한글 사용
			\usepackage{kotex}					% 한글 사용
			\usepackage[unicode]{hyperref}			% 한글 하이퍼링크 사용

		% ------------------------------ 수학 수식
			\usepackage{amssymb,amsfonts,amsmath}	% 수학 수식 사용
			\usepackage{mathtools}				% amsmath 확장판

			\usepackage{scrextend}				% 
		

		% ------------------------------ LIST
			\usepackage{enumerate}			%
			\usepackage{enumitem}			%
			\usepackage{tablists}				%	수학문제의 보기 등을 표현하는데 사용
										%	tabenum


		% ------------------------------ table 
			\usepackage{longtable}			%
			\usepackage{tabularx}			%
			\usepackage{tabu}				%




		% ------------------------------ 
			\usepackage{setspace}			%
			\usepackage{booktabs}		% table
			\usepackage{color}			%
			\usepackage{multirow}			%
			\usepackage{boxedminipage}	% 미니 페이지
			\usepackage[pdftex]{graphicx}	% 그림 사용
			\usepackage[final]{pdfpages}		% pdf 사용
			\usepackage{framed}			% pdf 사용

			
			\usepackage{fix-cm}	
			\usepackage[english]{babel}
	
		%	======================================================================================= package
		% 	tikz package
		% 	
		% 	--------------------------------- 	
			\usepackage{tikz}%
			\usetikzlibrary{arrows,positioning,shapes}
			\usetikzlibrary{mindmap}			
			

		% --------------------------------- 	page
			\usepackage{afterpage}		% 다음페이지가 나온면 어떻게 하라는 명령 정의 패키지
%			\usepackage{fullpage}			% 잘못 사용하면 다 흐트러짐 주의해서 사용
%			\usepackage{pdflscape}		% 
			\usepackage{lscape}			%	 


			\usepackage{blindtext}
	
		% --------------------------------- font 사용
			\usepackage{pifont}				%
			\usepackage{textcomp}
			\usepackage{gensymb}
			\usepackage{marvosym}



		% Package --------------------------------- 

			\usepackage{tablists}				%


		% Package --------------------------------- 
			\usepackage[framemethod=TikZ]{mdframed}				% md framed package
			\usepackage{smartdiagram}								% smart diagram package

			\newcounter{theo}[section]\setcounter{theo}{0}
			\renewcommand{\thetheo}{\arabic{section}.\arabic{theo}}
%			\newenvironment{name}[args]{begin_def}{end_def}

 
  


		% Package ---------------------------------    연습문제 

			\usepackage{exsheets}				%

			\SetupExSheets{solution/print=true}
			\SetupExSheets{question/type=exam}
			\SetupExSheets[points]{name=point,name-plural=points}


		% --------------------------------- 페이지 스타일 지정

		\usepackage[Sonny]		{fncychap}

			\makeatletter
			\ChNameVar	{\Large\bf}
			\ChNumVar	{\Huge\bf}
			\ChTitleVar		{\Large\bf}
			\ChRuleWidth	{0.5pt}
			\makeatother

%		\usepackage[Lenny]		{fncychap}
%		\usepackage[Glenn]		{fncychap}
%		\usepackage[Conny]		{fncychap}
%		\usepackage[Rejne]		{fncychap}
%		\usepackage[Bjarne]	{fncychap}
%		\usepackage[Bjornstrup]{fncychap}

		\usepackage{fancyhdr}
		\pagestyle{fancy}
		\fancyhead{} % clear all fields
		\fancyhead[LO]{\footnotesize \leftmark}
		\fancyhead[RE]{\footnotesize \leftmark}
		\fancyfoot{} % clear all fields
		\fancyfoot[LE,RO]{\large \thepage}
		%\fancyfoot[CO,CE]{\empty}
		\renewcommand{\headrulewidth}{1.0pt}
		\renewcommand{\footrulewidth}{0.4pt}
	
	
	
		%	--------------------------------------------------------------------------------------- 
		% 	tritlesec package
		% 	
		% 	
		% 	------------------------------------------------------------------ section 스타일 지정
	
			\usepackage{titlesec}
		
		% 	----------------------------------------------------------------- section 글자 모양 설정
			\titleformat*{\section}					{\large\bfseries}
			\titleformat*{\subsection}				{\normalsize\bfseries}
			\titleformat*{\subsubsection}			{\normalsize\bfseries}
			\titleformat*{\paragraph}				{\normalsize\bfseries}
			\titleformat*{\subparagraph}				{\normalsize\bfseries}
	
		% 	----------------------------------------------------------------- section 번호 설정
			\renewcommand{\thepart}				{\arabic{part}.}
			\renewcommand{\thesection}				{\arabic{section}.}
			\renewcommand{\thesubsection}			{\thesection\arabic{subsection}.}
			\renewcommand{\thesubsubsection}		{\thesubsection\arabic{subsubsection}}
			\renewcommand\theparagraph 			{$\blacksquare$ \hspace{3pt}}

		% 	----------------------------------------------------------------- section 페이지 나누기 설정
			\let\stdsection\section
			\renewcommand\section{\newpage\stdsection}



		%	--------------------------------------------------------------------------------------- 
		% 	\titlespacing*{commandi} {left} {before-sep} {after-sep} [right-sep]		
		% 	left
		%	before-sep		:  수직 전 간격
		% 	after-sep	 	:  수직으로 후 간격
		%	right-sep

			\titlespacing*{\section} 			{0pt}{1.0em}{1.0em}
			\titlespacing*{\subsection}	  		{0ex}{1.0em}{1.0em}
			\titlespacing*{\subsubsection}		{0ex}{1.0em}{1.0em}
			\titlespacing*{\paragraph}			{0em}{1.5em}{1.0em}
			\titlespacing*{\subparagraph}		{4em}{1.0em}{1.0em}
	
		%	\titlespacing*{\section} 			{0pt}{0.0\baselineskip}{0.0\baselineskip}
		%	\titlespacing*{\subsection}	  		{0ex}{0.0\baselineskip}{0.0\baselineskip}
		%	\titlespacing*{\subsubsection}		{6ex}{0.0\baselineskip}{0.0\baselineskip}
		%	\titlespacing*{\paragraph}			{6pt}{0.0\baselineskip}{0.0\baselineskip}
	

		% --------------------------------- recommend		섹션별 페이지 상단 여백
			\newcommand{\SectionMargin}				{\newpage  \null \vskip 2cm}
			\newcommand{\SubSectionMargin}			{\newpage  \null \vskip 2cm}
			\newcommand{\SubSubSectionMargin}		{\newpage  \null \vskip 2cm}


		%	--------------------------------------------------------------------------------------- 
		%
		% 	toc 설정  - table of contents
		% 	
		% 	
		% 	----------------------------------------------------------------  문서 기본 사항 설정
			\setcounter{secnumdepth}{0} 		% 문단 번호 깊이 chapter
			\setcounter{secnumdepth}{1} 		% 문단 번호 깊이 section
			\setcounter{secnumdepth}{2} 		% 문단 번호 깊이 subsection
			\setcounter{secnumdepth}{3} 		% 문단 번호 깊이 subsubsec
			\setcounter{secnumdepth}{4} 		% 문단 번호 깊이 paragraph
			\setcounter{secnumdepth}{5} 		% 문단 번호 깊이 subparagraph
			\setcounter{tocdepth}{2} 			% 문단 번호 깊이 - 목차 출력시 출력 범위

			\setlength{\parindent}{0cm} 		% 문서 들여 쓰기를 하지 않는다.


		%	--------------------------------------------------------------------------------------- 
		% 	mini toc 설정
		% 	
		% 	
		% 	--------------------------------------------------------- 장의 목차  minitoc package
			\usepackage{minitoc}

%			\setcounter{minitocdepth}{0}    	% 	chapter 
			\setcounter{minitocdepth}{1}    	%  	secton
%			\setcounter{minitocdepth}{2}    	%  	subsection
%			\setcounter{minitocdepth}{3}    	%  	sub sub section
%			\setcounter{minitocdepth}{4}    	%  	paragraph
%			\setcounter{minitocdepth}{5}    	%  	sub paragraph
%			\setlength{\mtcindent}{12pt} 		%	default 24pt
			\setlength{\mtcindent}{24pt} 		% 	default 24pt

		% 	--------------------------------------------------------- part toc
		%	\setcounter{parttocdepth}{2} 		%  default
			\setcounter{parttocdepth}{0}
		%	\setlength{\ptcindent}{0em}		%  default  목차 내용 들여 쓰기
			\setlength{\ptcindent}{0em}         


		% 	--------------------------------------------------------- section toc
			\renewcommand{\ptcfont}{\normalsize\rm} 		%  default
			\renewcommand{\ptcCfont}{\normalsize\bf} 	%  default
			\renewcommand{\ptcSfont}{\normalsize\rm} 	%  default

		% 	--------------------------------------------------------- section toc
%			\setcounter{sectiondepth}{2}
%			\setcounter{\secttocdepth}{2}				% default
%			\setlength{\stcindent}{24pt}				% default
%			\renewcommand{\stcfont}{f\small\rm}		% default
%			\renewcommand{\stcSSfont}{f\small\bf} 	% default


		%	=======================================================================================
		% 	tocloft package
		% 	
		% 	------------------------------------------ 목차의 목차 번호와 목차 사이의 간격 조정
			\usepackage{tocloft}

		% 	------------------------------------------ 목차의 내어쓰기 즉 왼쪽 마진 설정
			\setlength{\cftsecindent}{2em}			%  section

		% 	------------------------------------------ 목차의 목차 번호와 목차 사이의 간격 조정
			\setlength{\cftsecnumwidth}{2em}		%  section





		%	=======================================================================================
		% 	flowchart  package
		% 	
		% 	------------------------------------------ 목차의 목차 번호와 목차 사이의 간격 조정
			\usepackage{flowchart}
			\usetikzlibrary{arrows}

		%	======================================================================================= 
		% 	t color box
		% 	
		% 	------------------------------------------ 상자안에 강조 문자
			\usepackage{tcolorbox}
%			\usepackage[listings]{tcolorbox}
			\tcbuselibrary{raster}
			\tcbuselibrary{breakable}

		%	=======================================================================================  코드 입력
			\usepackage[]{quoting}
			\usepackage{csquotes}		%
			\usepackage{epigraph}		%
			\usepackage{epigraph}		%

			\usepackage{epigraph}		% 	 \begin{savequote}[45mm]
										%		\qauthor{Shakespeare, Macbeth}
										%		Cookies! Give me some cookies!
										%		\qauthor{Cookie Monster}
										%	\end{savequote}

										%	\begin{quote}
										%	\lipsum[1-2]
										%	\end{quote}

										%	\begin{quotation}
										%	\lipsum[1-2]
										%	\end{quotation}

										%	\begin{quoting}
										%	\lipsum[1-2]
										%	\end{quoting}

										%	\begin{verbatim}
										%		core.ignorecase=true
										%	\end{verbatim}



		%	=======================================================================================
		% 		makeindex package
		% 	
		% 	------------------------------------------ 목차의 목차 번호와 목차 사이의 간격 조정
%			\usepackage{makeindex}
%			\usepackage{makeidy}



		%	=======================================================================================
		% 		각주와 미주
		% 	

		\usepackage{endnotes} %미주 사용


		%	=======================================================================================
		% 	줄 간격 설정
		% 	
		% 	
		% 	--------------------------------- 	줄간격 설정
			\doublespace
%			\onehalfspace
%			\singlespace
		
		

	% 	============================================================================== itemi Global setting

	
		%	-------------------------------------------------------------------------------
		%		Vertical spacing
		%	-------------------------------------------------------------------------------
			\setlist[itemize]{topsep=0.0em}			% 상단의 여유치
			\setlist[itemize]{partopsep=0.0em}			% 
			\setlist[itemize]{parsep=0.0em}			% 
%			\setlist[itemize]{itemsep=0.0em}			% 
			\setlist[itemize]{noitemsep}				% 
			
		%	-------------------------------------------------------------------------------
		%		Horizontal spacing
		%	-------------------------------------------------------------------------------
			\setlist[itemize]{labelwidth=1em}			%  라벨의 표시 폭
			\setlist[itemize]{leftmargin=8em}			%  본문 까지의 왼쪽 여백  - 4em
			\setlist[itemize]{labelsep=3em} 			%  본문에서 라벨까지의 거리 -  3em
			\setlist[itemize]{rightmargin=0em}			% 오른쪽 여백  - 4em
			\setlist[itemize]{itemindent=0em} 			% 점 내민 거리 label sep 과 같은면 점위치 까지 내민다
			\setlist[itemize]{listparindent=3em}		% 본문 드려쓰기 간격
	
	
			\setlist[itemize]{ topsep=0.0em, 			%  상단의 여유치
						partopsep=0.0em, 		%  
						parsep=0.0em, 
						itemsep=0.0em, 
						labelwidth=1em, 
						leftmargin=2.5em,
						labelsep=2em,			%  본문에서 라벨 까지의 거리
						rightmargin=0em,		% 오른쪽 여백  - 4em
						itemindent=0em, 		% 점 내민 거리 label sep 과 같은면 점위치 까지 내민다
						listparindent=0em}		% 본문 드려쓰기 간격
	
%			\begin{itemize}
	
		%	-------------------------------------------------------------------------------
		%		Label
		%	-------------------------------------------------------------------------------
			\renewcommand{\labelitemi}{$\bullet$}
			\renewcommand{\labelitemii}{$\bullet$}
%			\renewcommand{\labelitemii}{$\cdot$}
			\renewcommand{\labelitemiii}{$\diamond$}
			\renewcommand{\labelitemiv}{$\ast$}		
	
%			\renewcommand{\labelitemi}{$\blacksquare$}   	% 사각형 - 찬것
%			\renewcommand\labelitemii{$\square$}		% 사각형 - 빈것	
			






% ------------------------------------------------------------------------------
% Begin document (Content goes below)
% ------------------------------------------------------------------------------ 표지
	\begin{document}
	
			\dominitoc
			\doparttoc			

%			\dosecttoc
%			\dosectlof
%			\dosectlot

			\title{ Git Gitlab  }
			\author{김대희}
			\date{ 	2020년 
					12월 
					30일
					수요일
						}
			\maketitle


			\tableofcontents 		% 목차 출력
%			\listoffigures 			% 그림 목차 출력
			\cleardoublepage
			\listoftables 			% 표 목차 출력





		\mdfdefinestyle	{con_specification} {
						outerlinewidth		=1pt			,%
						innerlinewidth		=2pt			,%
						outerlinecolor		=blue!70!black	,%
						innerlinecolor		=white 			,%
						roundcorner			=4pt			,%
						skipabove			=1em 			,%
						skipbelow			=1em 			,%
						leftmargin			=0em			,%
						rightmargin			=0em			,%
						innertopmargin		=2em 			,%
						innerbottommargin 	=2em 			,%
						innerleftmargin		=1em 			,%
						innerrightmargin		=1em 			,%
						backgroundcolor		=gray!4			,%
						frametitlerule		=true 			,%
						frametitlerulecolor	=white			,%
						frametitlebackgroundcolor=black		,%
						frametitleaboveskip=1em 			,%
						frametitlebelowskip=1em 			,%
						frametitlefontcolor=white 			,%
						}


%	개정일력
%	2020.12.30 첫 작업
%	2020.12.30 깃 업로드
%


%	================================================================== Part			개요
	\addtocontents{toc}{\protect\newpage}
	\part{ 개요  }
	\noptcrule
	\parttoc				

% ----------------------------------------------------------------------------- 	깃 시작하기
%										
% -----------------------------------------------------------------------------										
	\section{ 깃 시작하기 }


% ----------------------------------------------------------------------------- 	깃랩
%										
% -----------------------------------------------------------------------------										
	\section{ Gitlab }

	\paragraph{}
Gitlab은 Git의 원격 저장소 기능과 이슈 트래커 기능등을 제공하는 소프트웨어다. 설치형 Github라는 컨셉으로 시작된 프로젝트이기 때문에 Github와 비슷한 면이 많이 있다. 서비스 형 원격저장소를 운영하는 것에 대한 비용이 부담되거나, 소스코드의 보안이 중요한 프로젝트에게 적당하다. 


							

	\paragraph{Gitlab의 특징}


			\begin{itemize}
			\item 	설치형 버전관리 시스템 - 자신의 서버에 직접 설치해서 사용할 수 있다. 
			\item 	클라우드 버전 관리 시스템 - gitlab.com을 이용하면 서버 없이도 Gitlab의 기능을 이용할 수 있다. 10명 이하의 프로젝트는 무료로 사용할 수 있다. 
			\item 	Issue tracker 제공
			\item 	Git 원격 저장소 제공
			\item 	API 제공 
			\item 	Team, Group 기능 제공
			\end{itemize}






% ----------------------------------------------------------------------------- 	깃랩설치하기
%	
% -----------------------------------------------------------------------------	
	\chapter 	{깃랩 설치하기}



% ----------------------------------------------------------------------------- 	깃랩설치하기
%	
% -----------------------------------------------------------------------------	
	\section 	{깃랩 설치하기}


% ----------------------------------------------------------------------------- 	깃랩세팅
%	
% -----------------------------------------------------------------------------	
	\section 	{깃랩 세팅}

% ----------------------------------------------------------------------------- 	깃랩설치확인
%	
% -----------------------------------------------------------------------------	
	\section 	{깃랩 설치 확인}





	
%	================================================================== 		깃랩전체기능
	\addtocontents{toc}{\protect\newpage}
	\chapter {깃랩 전체 기능 }
	\noptcrule

%	\newpage	
%	\minitoc
%	\secttoc



% ----------------------------------------------------------------------------- 	저장소 생성
%	
% -----------------------------------------------------------------------------	
	\section 	{저장소 생성 :  git init}


% ----------------------------------------------------------------------------- 	git add
%	
% -----------------------------------------------------------------------------	
	\section 	{git add}


% ----------------------------------------------------------------------------- 	git commit
%	
% -----------------------------------------------------------------------------	
	\section 	{git commit}


% ----------------------------------------------------------------------------- 	git status
%	
% -----------------------------------------------------------------------------	
	\section 	{git status}


% ----------------------------------------------------------------------------- 	저장소 사용을 위한 명령
%	
% -----------------------------------------------------------------------------	
	\section 	{저장소 사용을 위한 명령}


%	================================================================== 		깃랩프로젝트
	\addtocontents{toc}{\protect\newpage}
	\chapter {Gitlab Project}
	\noptcrule

%	\newpage	
	\minitoc
%	\secttoc

% ----------------------------------------------------------------------------- 	 Git 기본 설정
%	
% -----------------------------------------------------------------------------	
	\section 	{ Git 기본 설정}


	각자의 이름과 이메일 주소를 적는다


		% ------------------------------------------------- tcolorbox package
		\begin{tcolorbox}		[
%								colback=green!5,
								colback=red!5!white,
								colframe=red!75!black,
%								colframe=green!40!black,
								title=깃 기본 설정
								]
			\$git config --glabal user.name  "kim dae hee 5609 " \\
			\$git config --glabal user.email  "h 010 3839 5609 @ gmail . com" \\
			\$git config --glabal color.ui true
		\end{tcolorbox}


		\begin{tcolorbox}
			- global : 지금 로그온한 계정에 전체 설정 \\
			- local : 현재 저장소 로컬 설정
		\end{tcolorbox}



%	================================================================== Part		깃랩이슈
	\addtocontents{toc}{\protect\newpage}
	\chapter 	{GitLab Issues}

	\noptcrule
	\minitoc
				

% ----------------------------------------------------------------------------- 	git ignore 란
%										
% -----------------------------------------------------------------------------										
	\section{git ignore 란}


		gitignore 파일이란 git버전관리에서 제외할 파일 목록을 지정하는 파일이다.

git으로 프로젝트를 관리할때, 그 프로젝트 안의 특정 파일을 관리할 필요가 없는 경우가 있다.
에를 들어 프로젝트 설정파일, 자동으로 생성되는 로그파일, 빌드인할때 생기는 컴파일 파일 등이 있다.
따라서 이러 관리할 필요가 없는 파일을 git이 track하지 않도록 .git ignore를 설정하는 것이다.











%	================================================================== Part			깃랩위키
	\addtocontents{toc}{\protect\newpage}
	\part 	{Gitlab wiki}


	\noptcrule
	\parttoc	
	\minitoc
				





%	================================================================== Part			팀그룹권한
	\addtocontents{toc}{\protect\newpage}
	\part 	{팀 그룹 권한 }
	\noptcrule
	\parttoc				

			





% ------------------------------------------------------------------------------
% End document
% ------------------------------------------------------------------------------
\end{document}


	\href{https://www.youtube.com/watch?v=SpqKCQZQBcc}{태양경배자세A}
	\href{https://www.youtube.com/watch?v=CL3czAIUDFY}{태양경배자세A}


https://docs.google.com/spreadsheets/d/1-wRuFU1OReWrtxkhaw9uh5mxouNYRP8YFgykMh2G_8c/edit#gid=0
+

https://seoyeongcokr-my.sharepoint.com/:f:/g/personal/02017_seoyoungeng_com/Ev8nnOI89D1LnYu90SGaVj0BTuckQ46vQe1HiVv-R4qeqQ?e=S3iAHi


		\begin{tcolorbox}[title=My title,
		colback=red!5!white,
		colframe=red!75!black,
		colbacktitle=yellow!50!red,
		coltitle=red!25!black,
		fonttitle=\bfseries,
		subtitle style={boxrule=0.4pt,
		colback=yellow!50!red!25!white} ]
		This is a \textbf{tcolorbox}.
		\tcbsubtitle{My subtitle}
		Further text.
		\tcbsubtitle{Second subtitle}
		Further text.
		\end{tcolorbox}
		

		\begin{tcbraster}[raster columns=3, raster equal height,
		size=small,colframe=red!50!black,colback=red!10!white,colbacktitle=red!50!white,
		title={Box \# \thetcbrasternum}]
			\begin{tcolorbox}First box\end{tcolorbox}
			\begin{tcolorbox}Second box\end{tcolorbox}
			\begin{tcolorbox}This is a box\\with a second line\end{tcolorbox}
			\begin{tcolorbox}Another box\end{tcolorbox}
			\begin{tcolorbox}A box again\end{tcolorbox}
		\end{tcbraster}