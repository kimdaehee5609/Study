%	------------------------------------------------------------------------------
%
%		공부 정리 순서 포스
%
%		2022년
%		08월
%		14일
%		토요일
%%		첫작업
%

%	\documentclass[25pt, a1paper]{tikzposter}
%	\documentclass[20pt, a0paper, landscape]{tikzposter}
%	\documentclass[25pt, a1paper ]{tikzposter}

	\documentclass[	20pt, 
							a1paper, 
%							portrait, %
							landscape,
							margin=0mm, %
							innermargin=10mm,  		%
							blockverticalspace=4mm, %
							colspace=5mm, 
							subcolspace=0mm
							]{tikzposter}

%	\documentclass[25pt, a1paper]{tikzposter}
%	\documentclass[25pt, a1paper]{tikzposter}
%	\documentclass[25pt, a1paper]{tikzposter}

% 	12pt  14pt 17pt  20pt  25pt
%
%	a0 a1 a2
%
%	landscape  portrait
%

	%% Tikzposter is highly customizable: please see
	%% https://bitbucket.org/surmann/tikzposter/downloads/styleguide.pdf

	%	========================================================== 	Package
		\usepackage{kotex}						% 한글 사용


%% Available themes: see also
%% https://bitbucket.org/surmann/tikzposter/downloads/themes.pdf
%	\usetheme{Default}
%	\usetheme{Rays}
%	\usetheme{Basic}
	\usetheme{Simple}
%	\usetheme{Envelope}
%	\usetheme{Wave}
%	\usetheme{Board}
%	\usetheme{Autumn}
%	\usetheme{Desert}

%% Further changes to the title etc is possible
%	\usetitlestyle{Default}			%
%	\usetitlestyle{Basic}				%
%	\usetitlestyle{Empty}				%
%	\usetitlestyle{Filled}				%
%	\usetitlestyle{Envelope}			%
%	\usetitlestyle{Wave}				%
%	\usetitlestyle{verticalShading}	%


%	\usebackgroundstyle{Default}
%	\usebackgroundstyle{Rays}
%	\usebackgroundstyle{VerticalGradation}
%	\usebackgroundstyle{BottomVerticalGradation}
%	\usebackgroundstyle{Empty}

%	\useblockstyle{Default}
%	\useblockstyle{Basic}
%	\useblockstyle{Minimal}		% 이것은 간단함
%	\useblockstyle{Envelope}		% 
%	\useblockstyle{Corner}		% 사각형
%	\useblockstyle{Slide}			%	띠모양  
	\useblockstyle{TornOut}		% 손그림모양


	\usenotestyle{Default}
%	\usenotestyle{Corner}
%	\usenotestyle{VerticalShading}
%	\usenotestyle{Sticky}

%	\usepackage{fontspec}
%	\setmainfont{FreeSerif}
%	\setsansfont{FreeSans}

%	------------------------------------------------------------------------------ 제목

	\title{공부 정리 순서 }

	\author{ 		작성 : 2022년 8월 14일 토요일  	수정 : 2022년 8월 24일 수요일 }

%	\institute{서영엔지니어링}
%	\titlegraphic{\includegraphics[width=7cm]{IMG_1934}}

	%% Optional title graphic
	%\titlegraphic{\includegraphics[width=7cm]{IMG_1934}}
	%% Uncomment to switch off tikzposter footer
	% \tikzposterlatexaffectionproofoff

\begin{document}
	\maketitle[
					width=841mm,
					linewidth = 2mm,
					innersep=4mm,
					titletotopverticalspace=2mm, %
					titletoblockverticalspace=2mm, %
					titletextscale =4, 
				]

		%		a0  841 - 1189
		%		a1  594 - 841
		%		a2  420 - 594

	\begin{columns}

	%	====== ====== ====== ====== ====== 01
		\column{0.2}

%	------------------------------------------------------------------------------ 토목
			\block{■  토목  }
			{
%					\setlength{\leftmargini}{4em}
%					\setlength{\labelsep} {1em}
					\begin{itemize}
					\item 구조    Structural
					\item 정역학 Statics
					\item 동역학  Dynamics
					\item 
					\item 강구조 설계           Steel-design
					\item 철근 콘크리트 구조 Reinforced-Concrete
					\item
					\item 토질        Soil
					\item 
					\item 수리		hydraulics
					\item 하천    	river-engineering
					\item 
					\item 도로        Road
					\item 도로배수  Road Drainer
					\item 
					\item 부대공   appurtenance / miscellanious work
					\item 
					\item 측량  Sutvey 
					\item 
					\item 시험   Experiment
					\item 
					\item 교량   Bridge
					\item 교대   Abut
					\item 교각   Pier
					\item 기초공학
					\item 내진설계 Seismic
					\item CAD
					\item BIM
					\end{itemize}
			} %	-----------------------------------------------------------------




	%	====== ====== ====== ====== ====== 02
		\column{0.2}


%	------------------------------------------------------------------------------ 자연 과학
			\block{■  자연 과학 }
			{
					\begin{itemize}
					\item 수학       Study-Math
					\item 물리       Study-Physics
					\item 화학       Study-Chemistry
					\item 천문학    Study-Astronomy
					\item 지학       Geo science
					\item 지리학    Geo graphy
					\item 광물학    mine rulogy
					\item 생명과학 Bological 
					\item 식물학    Botany
					\item 동물학    Zoology
					\end{itemize}
			} %	-----------------------------------------------------------------


%	------------------------------------------------------------------------------ 동물학
			\block{■  동물학 }
			{
					\begin{itemize}
					\item 곤충학 Insect
					\item 조류
					\item 
					\item 
					\item 
					\item 
					\item 
					\item 
					\item 
					\end{itemize}
			} %	-----------------------------------------------------------------





	%	====== ====== ====== ====== ======  03
		\column{0.2}

%	------------------------------------------------------------------------------ 캘리그라피
			\block{■  캘리그라피 }
			{
					\begin{itemize}
					\item 캘리그라피 Calligraphy
					\item 서예
					\item 
					\end{itemize}
			} %	-----------------------------------------------------------------


	%	====== ====== ====== ====== ======  04
		\column{0.2}

%	------------------------------------------------------------------------------ 역사
			\block{■  역사 }
			{
					\begin{itemize}
					\item 	역사    History
					\item 	박물관 Museum33 \\
							성보 박물관은 불교로 분류
					\item 	신화    Mythplogy
					\item 
					\item 
					\end{itemize}
			} %	-----------------------------------------------------------------






	%	====== ====== ====== ====== ======  05
		\column{0.2}

%	------------------------------------------------------------------------------ 
			\block{■    }
			{
				\begin{LARGE}
					\begin{itemize}
					\item 
					\item 
					\item 
					\item 
					\end{itemize}
				\end{LARGE}
			} %	-----------------------------------------------------------------

%	------------------------------------------------------------------------------  국민 내일 배움 카드
			\block{■  국민 내일 배움 카드  }
			{
				\begin{LARGE}
					\begin{itemize}
					\item 
					\item 
					\item 
					\item 
					\end{itemize}
				\end{LARGE}
			} %	-----------------------------------------------------------------


%	------------------------------------------------------------------------------ 파일
			\block{■  파일 }
			{
					\begin{itemize}
					\item 폴더명 : Study
					\item 파일명 : 공부 정리-포스트.tex
					\end{itemize}
			}






	\end{columns}




\end{document}


